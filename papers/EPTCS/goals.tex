\section{Streams, States, and Goals}
\label{sec:goals}

This section describes a top-level framework for our implementation. Even though it contains
nothing more than a reiteration of the original implementation~\cite{MicroKanren, CKanren}
in OCaml, we still need some notions to be properly established.

The search itself is implemented using a backtracking lazy stream monad\cite{KiselyovBacktracking}:

\begin{lstlisting}
   type $\alpha$ stream

   val mplus : $\alpha$ stream -> $\alpha$ stream -> $\alpha$ stream
   val bind  : $\alpha$ stream -> ($\alpha$ -> $\beta$ stream) -> $\beta$ stream
\end{lstlisting}

Monadic primitives describe the shape of the search, and their implementations may
vary in concrete \miniKanren versions.

An essential component of the implementation is a bundle of the following types:

\begin{lstlisting}
   type env         = $\dots$
   type subst       = $\dots$
   type constraints = $\dots$

   type state = env * subst * constraints
\end{lstlisting}

Type \lstinline{state} describes a point in a lazily constructed search tree: type \lstinline{env} corresponds
to an \emph{environment}, which contains some supplementary information (in particular, an environment is needed to
correctly allocate fresh variables), type \lstinline{subst} describes a substitution, which keeps current bindings
for some logical variables, and, finally, type \lstinline{constraints} represents disequality constraints,
which have to be respected. In the simplest case \lstinline{env} is just a counter for the number of the next free
variable, \lstinline{subst} is a map-like structure and \lstinline{constraints} is a list of substitutions.

In our case, the environment contains some extra information to make it possible to identify variables
in a constant time. This check is not needed in faster-miniKanren or any other untyped approach to implement
relational DSL. In case of \textit{Ocanren} it is dangerous to allow reification of a logic variable
(attributed by a type one the first state) in the context of an another state where it is attributed by another type.
Cases like this one can trigger undefined behavior. In the current implementation of logical variables created in the foreign
state are not treated as logic variables. This restriction is difficult to express in types, so the check was moved
from compile-time to runtime. This design decision is unfortunate, however this check doesn't affect perfomance in
any significant way and we don't know any useful program where such a mistake can happen.

The next cornerstone element is the \emph{goal} type, which is considered as a transformer of a state into
a lazy stream of states:

\begin{lstlisting}
   type goal = state -> state stream
\end{lstlisting}

In terms of the search, a goal nondeterministically performs one step of the search: for a given
node in a search tree it produces its immediate descendants. On the user level type \lstinline{goal}
is abstract, and states are completely hidden.

Next to last, there are a number of predefined combinators:

\begin{lstlisting}
   val (&&&)      : goal -> goal -> goal
   val (|||)      : goal -> goal -> goal
   val call_fresh : ($v$ -> goal) -> goal
   ....
\end{lstlisting}

Conjunction ``\lstinline{&&&}'' combines the results of its argument goals using \lstinline{bind},
disjunction ``\lstinline{|||}'' concatenates the results using \lstinline{mplus}, abstraction
primitive \lstinline{call_fresh} takes an abstracted goal and applies it to a freshly created
variable. Type $v$ in the last case designates the type for a fresh variable, which we leave
abstract for now. These combinators serve as the bricks for the implementation of conventional
\miniKanren constructs and syntax extensions (\lstinline{conde}, \lstinline{fresh}, etc.)

Finally, there are two primitive goal constructors:

\begin{lstlisting}
   val (===) : $t$ -> $t$ -> goal
   val (=/=) : $t$ -> $t$ -> goal
\end{lstlisting}

The first one is unification, while the other is the disequality constraint. Here, we again left
the type of terms $t$ abstract; it will be instantiated later.

In the implementation of \miniKanren both of these goals are implemented using unification~\cite{CKanren}; this
is true for us as well. However, due to a drastic difference in host languages, the implementation of
efficient polymorphic unification itself leads to a number of tricks with typing and data representation
absent in the original version.

In this setting, the run primitive is represented by the following function:

\begin{lstlisting}
   val run : goal -> state stream
\end{lstlisting}

This function creates an initial state and applies a goal to it. The states in the return stream describe
various solutions for the goal. As the stream is constructed lazily, taking elements one by one makes
the search progress.

To discover concrete answers, the state has to be queried for its contents. As a rule, a few variables
are reified in a state, i.e. their bindings in the corresponding substitution are retrieved.
Disequality constraints for free variables have to be reified additionally (e.g. represented as a list of
``forbidden'' terms). As forbidden terms can contain free variables, the constraint reification
process is recursive.

In our case, the reification is a subtle part, since, as we will see shortly, it can not be implemented in a
type-safe fragment of the language.
