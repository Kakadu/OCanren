\section{\miniKanren~--- a Short Presentation}
\label{sec:demo}

In this section we briefly describe \miniKanren in its original form, using a canonical example. 
\miniKanren is organized as a set of combinators and macros for Scheme/Racket, designated to describe 
a search for the solution of a certain \emph{goal}. There are four domain-specific constructs 
to build \emph{goals}:

\begin{itemize}
\item Syntactic unification~\cite{Unification} in the form \lstinline[language=scheme]{(== $t_1$ $t_2$)}, where $t_1$, $t_2$ are
some \emph{terms}; unification establishes a syntactic basis for all other goals. If there is a unifier for
two given terms, the goal is considered satisfied, a most general unifier is kept as a partial solution, and the execution
of current branch continues. Otherwise the current branch backtracks.

\item Disequality constraint~\cite{CKanren} in the form \lstinline[language=scheme]{($\not\equiv$ $t_1$ $t_2$)}, where
$t_1$, $t_2$ are some terms; a disequality constraint prevents all branches (starting from the current), in which the
specified terms are equal (w.r.t. the search state), from being considered.

\item Conditional construct in the form 

\begin{lstlisting}[language=scheme]
   (conde
      [$g^1_1\;\;g^1_2\;\;\dots\;\;g^1_{k_1}$]
      [$g^2_1\;\;g^2_2\;\;\dots\;\;g^2_{k_2}$]
      $\ldots$
      [$g^n_1\;\;g^n_2\;\;\dots\;\;g^n_{k_n}$]
   )
\end{lstlisting}

where each $g^i_j$ is a goal. A \lstinline{conde} goal considers each collection of subgoals, surrounded by square brackets, as
implicit conjunction (so \lstinline[language=scheme]{[$g^i_1\;\;g^i_2\;\;\dots\;\;g^i_{k_i}$] } is considered as a 
conjunction of all $g^i_j$) and tries to satisfy each of them independently~--- in other words, operates on them
as a disjunction.

\item Fresh variable introduction construct in the form

\begin{lstlisting}[language=scheme]
   (fresh ($x_1\;\;x_2\;\;\dots\;\;x_k$)
     $g_1$
     $g_2$
     $\ldots$
     $g_n$
   )
\end{lstlisting}

where each $g_i$ is a goal. This form introduces fresh variables $x_1\;\;x_2\;\;\dots\;\;x_k$ and
tries to satisfy the conjunction of all subgoals $g_1\;\;g_2\;\;\dots\;\;g_n$ (these subgoals may contain
introduced fresh variables).
\end{itemize}

As an example consider a list concatenation relation; by a well-established tradition, the names 
of relational objects are superscripted by ``$^o$'', hence \lstinline{append$^o$}: 

\begin{lstlisting}[mathescape=true,language=scheme,numbers=left,numberstyle=\small,stepnumber=1,numbersep=-5pt]
   (define (append$^o$ x y xy) 
      (conde 
         [(== '() x) (== y xy)]
         [(fresh (h t ty)
            (== `(,h . ,t) x)
            (== `(,h . ,ty) xy)
            (append$^o$ t y ty))]))
\end{lstlisting}

We interpret the relation ``\lstinline{append$^o$ x y xy}'' as ``the concatenation of \lstinline{x} and \lstinline{y} 
equals \lstinline{xy}''. Indeed, if the list \lstinline{x} is empty, then (regardless the content of \lstinline{y}) in order for the relation to hold 
the value for \lstinline{xy} should by equal to that of \lstinline{y}~--- hence line 3. Otherwise, \lstinline{x} can be decomposed into the head 
\lstinline{h} and the tail \lstinline{t}~--- so we need some fresh variables. We also need the additional variable \lstinline{ty} to designate the list 
that is in the relation \lstinline{append$^o$} with \lstinline{t} and \lstinline{y}. Trivial relational reasonings complete the implementation (lines 5-7).

A goal, built with the aid of the aforementioned constructs, can be run by the following primitive:

\begin{lstlisting}[mathescape=true,language=scheme]
   run $n$ ($q_1\dots q_k$) $G$
\end{lstlisting}

Here $n$ is the number of requested answers (or ``*'' for all answers), $q_i$ are fresh query variables, and $G$ is a goal, which can
contain these variables. 

The \lstinline{run} construct initiates the search for answers for a given goal and returns a (finite or infinite) list 
of answers~--- the bindings for query variables, which represent individual solutions for that query. For example,

\begin{lstlisting}[mathescape=true,language=scheme]
   run 1 (q) (append$^o$ '(1 2) '(3 4) q) 
\end{lstlisting}

\noindent returns a list \lstinline{((1 2 3 4))}, which constitutes the answer for query variable $q$. The process of constructing
the answers from internal data structures of miniKanren interpreter is called \emph{reification}~\cite{WillThesis}. 

