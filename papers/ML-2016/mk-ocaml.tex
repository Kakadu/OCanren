\documentclass[10pt, oneside, nocopyrightspace]{sigplanconf}

\usepackage{amssymb}
\usepackage{listings}
\usepackage{indentfirst}
\usepackage{verbatim}
\usepackage{amsmath, amsthm, amssymb}
\usepackage{graphicx}
\usepackage[hyphens]{url}
\usepackage[hidelinks]{hyperref}

\topmargin 2.0cm
\setlength{\textheight}{25.3cm}

\lstdefinelanguage{ocaml}{
keywords={let, begin, end, in, match, type, and, fun, function, try, with, when, class, 
object, method, of, rec, repeat, until, while, not, do, done, as, val, inherit, 
new, module, sig, deriving, datatype, struct, if, then, else, open, private, virtual},
sensitive=true,
basicstyle=\small,
commentstyle=\small\itshape\ttfamily,
keywordstyle=\ttfamily\underbar,
identifierstyle=\ttfamily,
basewidth={0.5em,0.5em},
columns=fixed,
fontadjust=true,
literate={->}{{$\;\;\to\;\;$}}1,
morecomment=[s]{(*}{*)}
}

\lstset{
basicstyle=\small,
identifierstyle=\ttfamily,
keywordstyle=\bfseries,
commentstyle=\scriptsize\rmfamily,
basewidth={0.5em,0.5em},
fontadjust=true,
escapechar=~,
language=ocaml
}

\sloppy

\newcommand{\miniKanren}{\texttt{miniKanren}}

\begin{document}

\title{Typed Embedding of Relational Language in OCaml}

\authorinfo{Dmitry Kosarev \and Dmitri Boulytchev}
{St.Petersburg State University \\ 
  Saint-Petersburg, Russia }
{$\mathtt{dboulytchev@gmail.com}$ \and $\mathtt{dboulytchev@math.spbu.ru}$}

\maketitle

\small
\section*{Introduction}

either to implement unification for each user type or to inject ...


\section{Logical Values}


\begin{lstlisting}[mathescape=true]
type $\alpha$ logic = private
| Var   of var * $\alpha$ logic list
| Value of $\alpha$
\end{lstlisting}

injection, projection, mapping and generic programming.

\section{Polymorphic Unification}

We consider it rather natural to employ polymorphic unification in the
language, already equipped with polymorphic comparison~--- a convenient, but
somewhat disputable\footnote{See, for example, 
             \href{https://blogs.janestreet.com/the-perils-of-polymorphic-compare}
                  {https://blogs.janestreet.com/the-perils-of-polymorphic-compare}
          } 
feature. Like polymorphic comparison, polymorphic unification performs traversal
of values, exploiting intrinsic knowledge of their runtime representation. 
The undeniable benefit of this solution is that in order to perform unification 
for user types no ``boilerplate'' code is needed. On the other hand, all pitfalls of
polymorphic comparison are inherited as well; in particular, unification can loop 
for cyclic data structures and does not work for functional values. Since we generally 
do not expect any reasonable outcome in these cases, the only remaining problem is that
the compiler 
%is completely oblivious to them and 
is uncapable to provide any assistance in identifying 
and avoiding them. Another drawback is that the implementation of polymorphic unification
relies on runtime representation of values and have to be fixed every time the representation changes. 
Finally, as it written in unsafe manner using \lstinline{Obj} interface, it has to be
carefully developed and tested.

An important difference between polymorphic comparison and unification is that the former 
only inspects its operands, while the results of unification are recorded in a substitution
(mapping from logical variables to terms), which later is used to refine answers and reify 
constraints. So, generally speaking, we have to show, that no ill-typed terms are constructed 
as a result.

Polymorphic unification is introduced via the following function:

\begin{lstlisting}[mathescape=true]
val unify : 
  $\alpha$ logic -> $\;\;\alpha$ logic -> $\;\;$subst option -> $\;\;$subst option
\end{lstlisting}

\noindent where ``\lstinline{subst}'' stands for the type of substitution. Unification can 
fail (hence ``\lstinline{option}'' in the result type), is performed in the context of
existing substitution (hence ``\lstinline{subst}'' in the third argument) and
can be chained (hence ``\lstinline{option}'' in the third argument). Note, that the 
type of substitution is not polymorphic, which means, that compiler competely looses the 
track of types for values, stored in a substitution. These types are recovered later during
refinement of answers.

To justify the correctness of unification, we consider the set of typed terms, each of which
has one of two forms

$$
x^\tau \mid C^\tau(t_1^{\tau_1},\dots,t_k^{\tau_k})
$$

\noindent where $x^\tau$ denotes a logical variable of type $\tau$, 
$C^\tau$~--- some constructor of type $\tau$, $t_i^{\tau_i}$~--- some terms of types $\tau_i$.
We reflect by $t_1^\tau[t_2^\rho]$ the fact of $t_2^\rho$ being a subterm of $t_1^\tau$, and
assume, that $\rho$ is unabiguously determined by $t_1$, $\tau$, and a position of $t_2$ 
``inside'' $t_1$.

Outside unification the compiler maintains typing, which means, that all 
terms, subterms, and variables agree in their types in all contexts. However, as 
our implementation resorts to unsafe features, we have to make this work for unification
code on our own.

We argue, that the following three invariants are maintained for any substitution $s$, involved 
in unification:

\begin{enumerate}
\item if \mbox{$t_1^{\_}[x^{\tau}]$} and \mbox{$t_2^{\_}[x^{\rho}]$}~--- two arbitrary terms (in particular, 
$t_1^{\_}$ and $t_2^{\_}$ may be the same), bound in $s$ and containing occurrences of variable $x$, 
then $\rho=\tau$ (different occurrences of the same variable in $s$ are attributed with the same type);

\item if \mbox{$(s\;\;x^\tau)$} is defined, then \mbox{$(s\;\;x^\tau) = t^\tau$} (a substitution always
binds a variable to a term of the same type);

\item each variable in $s$ preserves its type, assigned by the compiler (from the first two invariants 
it is follows, that this type is unique; note also, that all variables are created and have their 
types assigned outside unification, in a type-safe world).
\end{enumerate}

The initial (empty) substitution trivially fulfills these invariants; hence, it is sufficient
to show, that they are preserved by unification.

The following snippet presents the implementation of unification with triangular 
substitution in only a little bit more abstract form, than actual code (for example, 
``occurs check'' is omitted):

\begin{lstlisting}[mathescape=true,numbers=left,numberstyle=\tiny,stepnumber=1,numbersep=5pt]
let rec walk $s$ = function
| $x^\tau$ when $x\in dom(s)$ -> $\;\;$walk $s$ $(s\;\;x)^\tau$
| $t^\tau$ -> $\;\;t^\tau$

let rec unify $t_1^\tau$ $t_2^\tau$ = function
| None -> None
| Some $s$ as $sub$  ->
    match walk $s$ $t_1$, walk $s$ $t_2$ with
    | $x_1^\tau$, $x_2^\tau$ when $x_1$ = $x_2$ -> $\;\;sub$
    | $x_1^\tau$, $(t_2^\prime)^\tau$ -> $\;\;$Some ($s[x_1 \gets t_2^\prime]$)
    | $(t_1^\prime)^\tau$, $x_2^\tau$ -> $\;\;$Some ($s[x_2 \gets t_1^\prime]$)
    | $C^\tau(t_1^{\tau_1},\dots,t_k^{\tau_k})$, $C^\tau(p_1^{\tau_1},\dots,p_k^{\tau_k})$ -> 
        unify $t_k^{\tau_k}$ $p_k^{\tau_k}$$(..$(unify $t_2^{\tau_2}$ $p_2^{\tau_2}$ (unify $t_1^{\tau_1}$ $p_1^{\tau_1}$ $sub$))$..$)
    | $\_$, $\_$ -> $\;\;$None
\end{lstlisting}

Type annotations, included in the snippet above, can be justified by the following 
reasonings\footnote{We omit the verbal description of unification algorithm due to a
looming lack of space; the details can be found in~\cite{MicroKanren}.}:

\begin{enumerate}
\item Line 2: the type of \mbox{$(s\;\;x^\tau)$} is $\tau$ due to the invariant 2; hence, 
the type of \lstinline{walk} result coincides with the type of its second argument.

\item Line 9: the substitution is left unchanged, hence all invariants are preserved.

\item Line 10 (and, symmetrically, line 11): first, note, that \mbox{$(s\;\;x_1)$} is undefined
(otherwise \lstinline{walk} would not return $x_1$). Then, $x_1$ and $t_2^\prime$ have the
same type, which justifies the preservation of invariant 2. Finally, either \mbox{$x_1=t_1$}
(and, then, $\tau$ is the type of $x_1$, assigned by the compiler), or $x_1$ is retrieved
from $s$ with type $\tau$~--- both cases justify invariants 1 and 3. The same applies to 
the pair $t_2^\prime$ and $t_2$.

\end{enumerate}

\begin{thebibliography}{99}
\bibitem{TRS}
Daniel P. Friedman, William E.Byrd, Oleg Kiselyov. The Reasoned Schemer. The MIT
Press, 2005.

\bibitem{MicroKanren}
Jason Hemann, Daniel P. Friedman. $\mu$Kanren: A Minimal Core for Relational Programming //
Proceedings of the 2013 Workshop on Scheme and Functional Programming (Scheme '13).

\bibitem{CKanren}
Claire E. Alvis, Jeremiah J. Willcock, Kyle M. Carter, William E. Byrd, and Daniel P. Friedman.
cKanren: miniKanren with Constraints // 
Proceedings of the 2011 Workshop on Scheme and Functional Programming (Scheme '11).

\end{thebibliography}

\end{document}

