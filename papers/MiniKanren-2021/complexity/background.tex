\section{Background: syntax and operational semantics}
\label{sec:background}

In this section we recollect some known formal descriptions for \mK language that will be used as a basis for our development.
Specifically, we restate the syntax of the basic version of the language and the operational semantics for program evaluation.
The descriptions here are taken from~\cite{CertifiedSemantics} (with a few non-essential adjustments for presentation purposes) to make
the paper self-contained, more details and explanations can be found there.

\begin{figure}[t]
\centering
\[
\begin{array}{ccll}
  \mathcal{C} & = & \{C_i^{k_i}\} & \mbox{constructors with arities} \\
  \mathcal{T}_X & = & X \cup \{C_i^{k_i} (t_1, \dots, t_{k_i}) \mid t_j\in\mathcal{T}_X\} & \mbox{terms over the set of variables $X$} \\
  \mathcal{D} & = & \mathcal{T}_\emptyset & \mbox{ground terms}\\
  \mathcal{X} & = & \{ x, y, z, \dots \} & \mbox{syntactic variables} \\
  \mathcal{A} & = & \{ x^?, y^?, z^? \dots \} & \mbox{logic variables} \\
  \mathcal{R} & = & \{ R_i^{k_i}\} &\mbox{relational symbols with arities} \\[2mm]
  \mathcal{G} & = & \mathcal{T_X}\equiv\mathcal{T_X}   &  \mbox{equality} \\
              &   & \mathcal{G}\wedge\mathcal{G}     & \mbox{conjunction} \\
              &   & \mathcal{G}\vee\mathcal{G}       &\mbox{disjunction} \\
              &   & \mbox{\lstinline|fresh|}\;\mathcal{X}\;.\;\mathcal{G} & \mbox{fresh variable introduction} \\
              &   & R_i^{k_i} (t_1,\dots,t_{k_i}),\;t_j\in\mathcal{T_X} & \mbox{relational symbol invocation} \\[2mm]
  \mathcal{S} & = & \{R_i^{k_i} = \lambda\;x_1^i\dots x_{k_i}^i\,.\, g_i;\}\; g & \mbox{specification}
\end{array}
\]
\caption{The syntax of \mK}
\label{fig:syntax}
\end{figure}

The syntax of the basic version of \mK is shown in Fig.~\ref{fig:syntax}. 
All data is presented using terms $\mathcal{T}_X$ built from a fixed set of constructors $\mathcal{C}$ with known arities and variables
from a given set $X$.
We parameterize the terms with an alphabet of variables since in the semantic description we will need \emph{two} kinds of variables:
\emph{syntactic} variables $\mathcal{X}$, used for bindings in the definition, and \emph{logic} variables $\mathcal{A}$, which are
introduced and unified during the evaluation.

In this version of \mK there are five types of goals: unification of two terms, conjunction and disjunction of goals (the
``\lstinline|conde|'' operator from the canonical versions of \mK, split in the ыtwo for simplicity), fresh logic variable introduction, and
invocation of some relational definition. For the sake of brevity, in code snippets we abbreviate immediately nested ``\lstinline|fresh|''
constructs into the one, writing ``\lstinline|fresh $x$ $y$ $\dots$ . $g$|'' instead of
``\lstinline|fresh $x$ . fresh $y$ . $\dots$ $g$|''.

The \emph{specification} $\mathcal{S}$ consists of a set of relational definitions and a top-level goal.
A top-level goal represents a search procedure which returns a stream of substitutions for the free variables of the goal.
The language we consider is first-order, as goals can not be passed as parameters, returned or constructed at runtime.

During the evaluation of \mK program an environment, consisting of a substitution for logic variables and a counter of allocated logic
variables, is threaded through computation and updated in every unification and fresh variable introduction.
The substitution in the environment at a given point and a given branch of evaluation contains all the information about relations between
the logical variables at this point, we will later refer to the substitution in the environment at a given point as a ``current substitution''
in informal explanations.
The different branches are combined via \emph{interleaving search} procedure~\cite{InterleavingSearch}.
The answers for the given query are extracted from the final environments (they are the values of the queried variables in the substitution
in the final environment).

This search procedure is formally described by an operational semantics in the form of a labeled transition system.
This semantics corresponds to the canonical implementation of interleaving search. 

\begin{figure}[t]
\centering
\[
\begin{array}{ccll}
  \Sigma & = & \mathcal{A} \to \mathcal{T}_\mathcal{A} & \mbox{substitutions} \\
  E & = & \Sigma \times \mathbb{N} & \mbox{environments} \\
  \\
  S & = & \taskst{\mathcal{G}}{E} & \mbox{task} \\
    &   & S \oplus S & \mbox{sum} \\
    &   & S \otimes \mathcal{G} & \mbox{product} \\
  \hat{S} & = & \diamond \; \mid \; S & \mbox{states} \\
  L & = & \circ \; \mid \; E & \mbox{labels} \\
\end{array}
\]
\caption{States and labels in the LTS for \mK}
\label{fig:operanional_semantics_states_labels}
\end{figure}

The form of states and labels in the transition system is defined in \figureword~\ref{fig:operanional_semantics_states_labels}.
Non-terminal states $S$ have a tree-like structure with intermediate nodes corresponding to partially-evaluated conjunctions
(``$\otimes$'') or disjunctions (``$\oplus$'').
A leaf in the form $\taskst{g}{e}$ determines a task to evaluate a goal $g$ in an environment $e$. For a conjunction node, its right child
is always a goal since it cannot be evaluated unless some result is provided by the left conjunct.
We also need a terminal state $\diamond$ to represent the end of the evaluation.
The label ``$\circ$'' is used to mark those steps which do not provide an answer; otherwise, a transition is labeled by an updated
environment.

\begin{figure*}
  \renewcommand{\arraystretch}{1.6}
  \[
  \begin{array}{cr}
    \taskst{t_1 \equiv t_2}{(\sigma, n)} \xrightarrow{\circ} \Diamond , \, \, \nexists\; mgu\,(t_1 \sigma, t_2 \sigma) &\ruleno{UnifyFail} \\
    \taskst{t_1 \equiv t_2}{(\sigma, n)} \xrightarrow{(mgu\,(t_1 \sigma, t_2 \sigma) \circ \sigma),\, n)} \Diamond & \ruleno{UnifySuccess} \\
    \taskst{g_1 \lor g_2}{e} \xrightarrow{\circ} \taskst{g_1}{e} \oplus \taskst{g_2}{e} & \ruleno{Disj} \\
    \taskst{g_1 \land g_2}{e} \xrightarrow{\circ} \taskst{g_1}{e} \otimes g_2 & \ruleno{Conj} \\
    \taskst{\mbox{\lstinline|fresh|}\, x\, .\, g}{(\sigma, n)} \xrightarrow{\circ} \taskst{g\,[\bigslant{\alpha_{n + 1}}{x}]}{( \sigma, n + 1)} & \ruleno{Fresh} \\
    \dfrac{R_i^{k_i}=\lambda\,x_1\dots x_{k_i}\,.\,g}{\taskst{R_i^{k_i}\,(t_1,\dots,t_{k_i})}{e} \xrightarrow{\circ} \taskst{g\,[\bigslant{t_1}{x_1}\dots\bigslant{t_{k_i}}{x_{k_i}}]}{e}} & \ruleno{Invoke}\\
    \dfrac{s_1 \xrightarrow{\circ} \Diamond}{(s_1 \oplus s_2) \xrightarrow{\circ} s_2} & \ruleno{DisjStop}\\
    \dfrac{s_1 \xrightarrow{e} \Diamond}{(s_1 \oplus s_2) \xrightarrow{e} s_2} & \ruleno{DisjStopAns}\\
    \dfrac{s \xrightarrow{\circ} \Diamond}{(s \otimes g) \xrightarrow{\circ} \Diamond} &\ruleno{ConjStop}\\
    \dfrac{s \xrightarrow{e} \Diamond}{(s \otimes g) \xrightarrow{\circ} \taskst{g}{e}}  & \ruleno{ConjStopAns}\\
    \dfrac{s_1 \xrightarrow{\circ} s'_1}{(s_1 \oplus s_2) \xrightarrow{\circ} (s_2 \oplus s'_1)} &\ruleno{DisjStep}\\
    \dfrac{s_1 \xrightarrow{e} s'_1}{(s_1 \oplus s_2) \xrightarrow{e} (s_2 \oplus s'_1)} &\ruleno{DisjStepAns}\\
    \dfrac{s \xrightarrow{\circ} s'}{(s \otimes g) \xrightarrow{\circ} (s' \otimes g)} &\ruleno{ConjStep}\\
    \dfrac{s \xrightarrow{e} s'}{(s \otimes g) \xrightarrow{\circ} (\taskst{g}{e} \oplus (s' \otimes g))} & \ruleno{ConjStepAns} 
  \end{array}
  \]
  \caption{Operational semantics of interleaving search}
  \label{fig:operanional_semantics_rules}
\end{figure*}

The transition rules are shown in Fig.~\ref{fig:operanional_semantics_rules}.
The introduced transition system is completely deterministic.
Therefore a derivation sequence for a state $s$ determines a certain \emph{trace}~--- a sequence of states and labeled transitions between
them.
It may be either finite (ending with the terminal state $\Diamond$) or infinite.
We will denote by $\trs{s}$ the sequence of states in the trace for initial state $s$ and by $\tra{s}$ the sequence of answers on
transitions in the trace for initial state $s$.
The sequence $\tra{s}$ corresponds to the stream of answers in the reference \mK implementations.
