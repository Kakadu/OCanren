\section{Scheduling Complexity}
\label{sec:scheduling}

We may notice that operational semantics described in the previous section can be used to calculate the number of elementary scheduling steps.
In this section we define a specific value which estimates the scheduling time and give some equations to calculate this value for a given \emph{semantic
state}. However, our ultimate goal is to provide a way to deduce the complexity factor recursively for a given query. This problem will be addressed in
Section~\ref{sec:symbolic}, which will make use of recurrent equations presented here.

We also restrict our considerations only by the case when the evaluation of a goal in question terminates. Indeed,
if the search diverges, no reasonable complexity estimation for the time of the whole search can be given. A much more interesting question would be
the complexity estimation for coming up with some \emph{specific} answer; however for now this problem seems to be too hard to
tackle.

Our first idea is to take the number of states $d\,(s)$ in the \emph{finite} trace for a given state $s$:

\[ d\,(s) \; \eqdef \; | \trs{s} |  \]

However, it turns out, that this value alone in not enough to provide an accurate scheduling complexity estimation. The reason is that some
elementary steps in the semantics are not elementary in existing implementations. Namely, a careful analysis of the semantics discovers that
it involves a navigation to the leftmost leaf of the state, which in an implementation corresponds to a number of
steps, proportional to the length of the leftmost branch of the state in question. Here we provide an \emph{ad-hoc} definition for this value, $t\,(s)$, which we call \emph{scheduling factor}:

\[
t\,(s) \eqdef \sum\limits_{s_i \in \trs{s}} lh\,(s_i) 
\]

where

\[
\begin{array}{rcl}
 lh\,(\taskst{g}{e})  &\eqdef& 1 \\
 lh\,(s_1 \oplus s_2) &\eqdef& lh\,(s_1) + 1 \\
 lh\,(s \otimes g)    &\eqdef& lh\,(s) + 1 \\
\end{array}
\]


The following lemma provides a fundamental relation between these two estimations of the scheduling complexity:

\begin{lemma}
  For any state $s$

  \[
  d\,(s) \le t\,(s) \le d^2\,(s)
  \]
  
\end{lemma}

Our next goal is to come up with the equations which relate the scheduling complexity of states with the scheduling complexity of their
(immediate) substates. We take schuduling factor $t\,(s)$ as a value that determines the scheduling complexity $T_s$, but we will also need to calculate $d\,(s)$ as it will be used in the equations for $t\,(s)$. 

The following lemma, obvious from the definitions, will be enough to deal with a basic (leaf state) case:

\begin{lemma}
  If

  \[\taskst{g}{e} \rightarrow s^\prime\]

  or

  \[\taskst{g}{e} \xrightarrow{a} s^\prime\]

  then

  \[d\,(\taskst{g}{e}) = d\,(s^\prime) + 1\]

  and

  \[t\,(\taskst{g}{e}) = t\,(s^\prime) + 1\]
\end{lemma}

In $\oplus$-states the substates are evaluated separately, one step at a time for each substate, so the total number of semantic steps is just a sum. However, for the scheduling factor there is an addition since the heights of the states in the trace become bigger (by one additional $\oplus$-node on the top). This additional node exists in the trace until one of the substates is evaluated completely, so the scheduling factor is increased by the number of steps before such event. So we have the following lemma.

\begin{lemma}
\label{lem:sum_estimation}
For any two states $s_1$ and $s_2$

\[
\begin{array}{rcl}
  d\,(s_1 \oplus s_2) &=& d\,(s_1) + d\,(s_2) \\

  t\,(s_1 \oplus s_2) &=& t\,(s_1) + t\,(s_2) + \min\,\{2\cdot d\,(s_1) - 1, 2\cdot d\,(s_2)\}
\end{array}
\]

\end{lemma}

For $\otimes$-states the reasoning is the same, but the result is more complicated. In $\otimes$-state the left substate is evaluated until the answer is found, which is then taken as \emph{an environment} for evaluation of the right subgoal. Thus, in the equations for $\otimes$-states answers
has to be taken into account. Since the tasks of evaluating the right subgoal in different environments are added to the evaluation of the left substate by creation of an $\oplus$-state we can use the equtions from the previous lemma. However, to do it we need calculate the number of steps in the evaluation of the left substate from the point a given answer is produces to the end. There is also an addition to the scheduling factor because of the $\otimes$-node that increases the height in the trace, analogous to the one caused by $\oplus$-node. We can notice that $\otimes$-node is always placed immediately over the left substate so this addition is exactely the number of steps for the left substate. Therefore the resulting equations for a $\otimes$-state are as follows.

\begin{lemma}
For any state $s$ and any goal $g$

\[
d\,(s \otimes g) = d\,(s) + \displaystyle\sum\limits_{a_i \in \tra{s}} d\,(\taskst{g}{a_i})
\]

\[
\begin{array}{rclc}
  t\,(s \otimes g) &=& t\,(s) & + \\
                   & & d\,(s) & + \\
  & & \displaystyle\sum\limits_{a_i \in \tra{s}} (t\,(\taskst{g}{a_i}) + \min\,\{2\cdot d\,(\taskst{g}{a_i}) - 1,\,2\cdot (d\,(s) - d_{a_i}\,(s) + \displaystyle\sum\limits_{j > i} d\,(\taskst{g}{a_j}))\})
\end{array}
\]

where $d_{a_i}\,(s)$ is the number of steps in the trace for state $s$ until the answer $a_i$ is produced.
\end{lemma}

The last equation is too cumbersome to use directly, so we will use some approximation of it instead. One option is to go with the fist argument of ``$\min$''. It should be a good approximation
in the case when there are several answers passed to the second goal and for none of them the number of steps surpass the \emph{overall} number of steps for all other answers (which is approximately the sum in the second argument of ``$\min$'').

\begin{corollary}
\label{lem:prod_estimation_multiple_answers}
For any state $s$ and any goal $g$
\[ t\,(s \otimes g) \le t\,(s) + d\,(s) +  \displaystyle\sum\limits_{a_i \in \tra{s}} (t\,(\taskst{g}{a_i}) + 2\cdot d\,(\taskst{g}{a_i}) \]
\end{corollary}

In the case when there is only one answer, however, we should rather go with the second argument of ``$\min$''.

\begin{corollary}
\label{lem:prod_estimation_single_answer}
  For any state $s$ and any goal $g$, if $\tra{s} = \{a\}$, then
  
\[ t\,(s \otimes g) \le t\,(s) + 3\cdot d\,(s) + t\,(\taskst{g}{a}) \]
\end{corollary}

Finally, since we will estimates only up to a multiplicative constant (in particular, it does not matter what constants we multiply values of $d(\cdot)$ by when calculating the scheduling factor) we can derive from these results two compact approximations for goals in the form of sequences of disjuncts/conjuncts. These two approximations work regardless the associativity/grouping of subformulas; thus a certain constant $c_k$, depending only on $k$, comes in.

For conjunctions we have the following one.

\begin{lemma}
\label{lem:conjunction_metrics_calc}

Let $g = g_1 \land \dots \land g_k$ and let $A_i$ be a set of all answers that are passed to $g_i$ at some stage starting from some initial environment $e_0$

\[
\begin{array}{rcl}
A_1 &=& \{ e_0 \} \\
A_{i + 1} & = & \bigcup\limits_{a \in A_i} \tra{\taskst{g_i}{a}} 
\end{array}
\]

Then

\[
\begin{array}{rcl}
d\,(\taskst{g}{e}) &=& \displaystyle\sum\limits_{1 \le i \le k} \;\; \displaystyle\sum\limits_{a \in A_i} d\,(\taskst{g_i}{a}) \\
t\,(\taskst{g}{e}) &\le& \displaystyle\sum\limits_{1 \le i \le k} \;\; \displaystyle\sum\limits_{a \in A_i} t\,(\taskst{g_i}{a}) + c_k \cdot \displaystyle\sum\limits_{1 \le i \le k} \;\; \displaystyle\sum\limits_{a \in A_i} d\,(\taskst{g_i}{a}), \\
\end{array}
\]

In the case when all $A_i$ contain only one answer

\[
t\,(\taskst{g}{e}) \le \displaystyle\sum\limits_{1 \le i \le k} \;\; \displaystyle\sum\limits_{a \in A_i} t\,(\taskst{g_i}{a}) + c_k \cdot \displaystyle \sum\limits_{1 \le i \le k - 1} \;\; \displaystyle\sum\limits_{a \in A_i} d\,(\taskst{g_i}{a})
\]

\end{lemma}

When applying the estimation from corollary~\ref{lem:prod_estimation_multiple_answers} we have an addition in form of number of steps (multiplied by some constant) for all conjuncts. The only exception is the case when every conjunct produces no more then one answer, then we can use the corollary~\ref{lem:prod_estimation_single_answer} everywhere instead and loose the additional number of steps for the last conjunct. We also have an additional constant number of steps turning conjunctions into $\otimes$-states but we can take it into account by choosing $c_k$.

For disjunctions the lemma is the following one.

\begin{lemma}
\label{lem:disjunction_metrics_calc}

Let $g = g_1 \lor \dots \lor g_k$ and $1 \le l \le k$; then

\[
\renewcommand{\arraystretch}{1.5}
\begin{array}{rcl}
  d\,(\taskst{g}{e}) &\le& \displaystyle\sum\limits_{1 \le i \le k} d\,(\taskst{g_i}{e}) \\
  t\,(\taskst{g}{e}) &\le& \displaystyle\sum\limits_{1 \le i \le k} t\,(\taskst{g_i}{e}) + c_k\cdot \displaystyle\sum\limits_{\renewcommand{\arraystretch}{1}\begin{array}{c}1 \le i \le k \\ i \ne l\end{array}} d\,(\taskst{g_i}{e}).
\end{array}
\]

\end{lemma}

Roughly speaking, if we have a disjunct $g_m$ with a number of steps bigger than all other disjuncts combined, then when applying lemma~\ref{lem:sum_estimation} we again will have an addition in form of number of steps for all disjuncts except for $g_m$ (it will allways be a bigger argument of ``$\min$''). But if we can lose the biggest number of steps among disjuncts, we can also lose any other one instead, so we allow to chose arbitrary $l$. The case when there is no disjunct with a number of steps bigger than all others should be considered separately (it is more obvious since then all the numbers of steps are the same up to a multiplicative constant).

